\documentclass[a4paper]{article}
\usepackage[OT1, OT2]{fontenc}
\setlength{\textheight}{25cm}
\setlength{\textwidth}{18cm}
\setlength{\topmargin}{-25mm}
\setlength{\hoffset}{-25mm}
\def\zn{,\kern-0.09em,}

\newcommand{\Lat}{\fontencoding{OT1}\fontfamily{cmr}\selectfont}

\begin{document}
\thispagestyle{empty}

\fontfamily{wncyr}
\fontencoding{OT2}\selectfont

\begin{flushleft}
Matematichki fakultet\\
Univerziteta u Beogradu
\end{flushleft}

\bigskip

\begin{center}
\textbf{MOLBA\\
ZA ODOBRAVANJE TEME MASTER RADA
}\end{center}

\bigskip

\begin{flushleft}
Molim da se odobri izrada master rada pod naslovom:
\end{flushleft}

\begin{minipage}{16.5cm}
%%%%%%%%%%%%%%%%%%%%%%%%%%%%%%%%%%%%%%%%%%%%%%%%%%%%%%%%%%%%%%%%%%%%%%%%%%%%%%%
% U donji red upisati naziv master rada umesto teksta: "Naziv master rada"    %
%%%%%%%%%%%%%%%%%%%%%%%%%%%%%%%%%%%%%%%%%%%%%%%%%%%%%%%%%%%%%%%%%%%%%%%%%%%%%%%
\textbf{\textit{\zn Implementacija dodatka za softver {\Lat Annovar} za prikaz funkcije i fenotipa gena''}}
\end{minipage}\\
\rule[4mm]{17.5cm}{.05mm}
\begin{flushleft}
\framebox{
\begin{minipage}[t][11cm]{17cm}
%%%%%%%%%%%%%%%%%%%%%%%%%%%%%%%%%%%%%%%%%%%%%%%%%%%%%%%%%%%%%%%%%%%%%%%%%%%%%%%
%  -- unutrasnjost pravougaonika --                                           %
%  Umesto donjeg teksta upisati znacaj i specificni cilj master rada          %
%%%%%%%%%%%%%%%%%%%%%%%%%%%%%%%%%%%%%%%%%%%%%%%%%%%%%%%%%%%%%%%%%%%%%%%%%%%%%%%
\textbf{Znachaj teme i oblasti:}\\
U genima kod razlichitih jedinki iste vrste, na nivou populacije, mozhe doc1i do razlika u nukleotidima na istim pozicijama. Ove razlike se nazivaju gene{t}ske varijacije.
Softver {\Lat Annovar} anotira genet{s}ke varijacije. Pod anotacijom se podrazumeva
pridruzhivanje relevantnih informacija svakoj varijaciji na osnovu razlichitih baza
podataka i na osnovu razlichitih alata, na primer na kojoj se poziciji na hromozomu
nalazi varijacija u odnosu na referentni genom, u kom je genu, kakvi su njeni predvidjeni funkcionalni efekti itd.

\textbf{Specifichni cilj rada:}\\
Cilj ovog rada je implementacija dodatka za softver {\Lat Annovar} koji bi omoguc1io prikaz informacija o funkciji i fenotipu gena. Ulazni podaci za {\Lat Annovar} su {\Lat .vcf} datoteke u kojima se nalaze informacije o varijacijama gena. Konkretno, u radu koristimo datoteke sa podacima \textit{Ashkenazi otac-majka-sin} trija iz projekta \zn Lichni genom''. Izlazni podaci su pridruzhene informacije, zapravo datoteke sa anotacijama. Iz izlazne datoteke {\Lat Annovar-a} se uzima informacija o genu na kojoj se nalazi odgovarajuc1a varijacija. Cilj plugin-a je da se, pored informacija koje se dobijaju korishc1enjem {\Lat Annovar}-a, prikazhu i funkcija i fenotip i to na osnovu dve baze znanja tj. ontologije - {\Lat \textit{Gene Ontology (GO)}} i {\Lat \textit{Human Phenotype Ontology (HPO)}}. Kako se datoteke za anotacije redovno azhuriraju, cilj je i ugradjivanje opcije za njihovo automatsko preuzimanje sa odgovarajucih lokacija na vebu.

\textbf{Literatura:}\\
Knjiga u kojoj su dati pojmovi vezani za varijacije \\
{\Lat Wing-Kin Sung, \textit{Algorithms for Next-Generation Sequencing},
CRC Press, Taylor and Francis Group, 2017}.\\
Zvanichna strana softvera {\Lat Annovar -  
http://annovar.openbioinformatics.org} \\
Ulazne {\Lat .vcf} datoteke - {\Lat ftp://ftp-trace.ncbi.nlm.nih.gov/giab/ftp/release/AshkenazimTrio} \\
Ontologija {\Lat Gene ontology - http://geneontology.org/gene-associations/goa\_human.gaf.gz} \\
Ontologija {\Lat Human Phenotype Ontology -
http://compbio.charite.de/jenkins/job/hpo.annotations.monthly/ \\
lastSuccessfulBuild/artifact/annotation/ALL\_SOURCES\_ALL\_FREQUENCIES\_genes\_to\_phenotype.txt }\\

%\begin{tabular}{|c|c|}
%    \hline
%    \multicolumn{2}{|c|}{\textbf{Uput{}stvo za pisanje nashih slova}} \\
%    \hline\hline
%	ligatura & rezultujuc1i simbol  \\
%    \hline
%    \texttt{{\Lat dj}} & dj \\
%    \hline
%    \texttt{{\Lat Dj}} & Dj \\
%    \hline
%    \texttt{{\Lat zh}} & zh \\

%    \hline
%    \texttt{{\Lat Zh}} & Zh \\
%    \hline
%    \texttt{{\Lat lj}} & lj \\
%    \hline
%    \texttt{{\Lat Lj}} & Lj \\
%    \hline
%    \texttt{{\Lat nj}} & nj \\
%    \hline
%    \texttt{{\Lat Nj}} & Nj \\
%    \hline
%    \texttt{{\Lat c1}} & c1 \\
%    \hline
%    \texttt{{\Lat C1}} & C1 \\
%    \hline
%    \texttt{{\Lat ch}} & ch \\
%    \hline
%    \texttt{{\Lat Ch}} & Ch \\
%    \hline
%    \texttt{{\Lat d2}} & d2 \\
%    \hline
%    \texttt{{\Lat D2}} & D2 \\
%    \hline
%    \texttt{{\Lat sh}} & sh \\
%    \hline
%    \texttt{{\Lat Sh}} & Sh \\
%    \hline
%    \texttt{{\Lat ts}} & ts \\
%    \hline
%    \texttt{{\Lat t\{\}s}} & t{}s \\
%    \hline
%\end{tabular}


\end{minipage}
}
\end{flushleft}
\vspace{0.5cm}
%%%%%%%%%%%%%%%%%%%%%%%%%%%%%%%%%%%%%%%%%%%%%%%%%%%%%%%%%%%%%%%%%%%%%%%%%%%%%%%
% u donji red uneti:         ime i prezime, broj indeksa i modul studenta     %
%%%%%%%%%%%%%%%%%%%%%%%%%%%%%%%%%%%%%%%%%%%%%%%%%%%%%%%%%%%%%%%%%%%%%%%%%%%%%%%
\makebox[7cm][c]{\textbf{Andjela Mijailovic1, 1050/2017, MR}}
%%%%%%%%%%%%%%%%%%%%%%%%%%%%%%%%%%%%%%%%%%%%%%%%%%%%%%%%%%%%%%%%%%%%%%%%%%%%%%%
% u donji red uneti:               ime i prezime mentora                      %
%%%%%%%%%%%%%%%%%%%%%%%%%%%%%%%%%%%%%%%%%%%%%%%%%%%%%%%%%%%%%%%%%%%%%%%%%%%%%%%
\hspace{1.7cm} Saglasan mentor \makebox[7cm][c]{\textbf{dr Jovana Kovachevic1, docent}} \\
\rule[4mm]{8.5cm}{.05mm} \hfill \raisebox{4mm}{\makebox[6cm][l]{.\dotfill.}} \\
\raisebox{1cm}%
[9mm][0mm]{\makebox[8cm][c]{\textit{(ime i prezime studenta, br. indeksa, smer i modul)}}} \\
\makebox[10cm]{ }\\

\vspace{-1cm}\\
\rule[2cm]{6.5cm}{.05mm} \hfill \rule[2cm]{6.5cm}{.05mm}\\
\vspace{-2.4cm}\\
\raisebox{2cm}{\makebox[6.5cm][c]{\textit{(svojeruchni potpis studenta)}}}
\hfill \raisebox{2cm}{\makebox[6.5cm][c]{\textit{(svojeruchni potpis mentora)}}}\\
\vspace{-2cm}\\
%%%%%%%%%%%%%%%%%%%%%%%%%%%%%%%%%%%%%%%%%%%%%%%%%%%%%%%%%%%%%%%%%%%%%%%%%%%%%%%
% u donji red uneti datum podnosenja molbe                                    %
%%%%%%%%%%%%%%%%%%%%%%%%%%%%%%%%%%%%%%%%%%%%%%%%%%%%%%%%%%%%%%%%%%%%%%%%%%%%%%%
\makebox[5.5cm][c]{}\makebox[5.5cm]{} Chlanovi komisije\\
%%%%%%%%%%%%%%%%%%%%%%%%%%%%%%%%%%%%%%%%%%%%%%%%%%%%%%%%%%%%%%%%%%%%%%%%%%%%%%%
% POPUNJAVA MENTOR (rucno ili na sledeci nacin):                              %
% u donji red umesto .dotfill. upisati podatke o 1. clanu komisije            %
%%%%%%%%%%%%%%%%%%%%%%%%%%%%%%%%%%%%%%%%%%%%%%%%%%%%%%%%%%%%%%%%%%%%%%%%%%%%%%%
\rule[4mm]{5.5cm}{.05mm}\makebox[5.5cm]{ } 1. \makebox[6cm][l]{dr Gordana Pavlovic1-Lazhetic1}\\%, redovni profesor}\\
\vspace{-8mm}\\
\raisebox{4mm}%
[7mm][0mm]{\makebox[5.5cm][c]{\textit{(datum podnoshenja molbe)}}}\makebox[5.5cm]{ }
%%%%%%%%%%%%%%%%%%%%%%%%%%%%%%%%%%%%%%%%%%%%%%%%%%%%%%%%%%%%%%%%%%%%%%%%%%%%%%%
% POPUNJAVA MENTOR (rucno ili na sledeci nacin):                              %
% u donji red umesto .\dotfill. upisati podatke o 2. clanu komisije           %
%%%%%%%%%%%%%%%%%%%%%%%%%%%%%%%%%%%%%%%%%%%%%%%%%%%%%%%%%%%%%%%%%%%%%%%%%%%%%%%
2. \makebox[6cm][l]{dr Nevena Veljkovic1} %nauchni savetnik INN \zn Vincha''}\\

\vspace{1cm}


\begin{flushleft}
%%%%%%%%%%%%%%%%%%%%%%%%%%%%%%%%%%%%%%%%%%%%%%%%%%%%%%%%%%%%%%%%%%%%%%%%%%%%%%%
% u donji red upisati                 katedru                                 %
%%%%%%%%%%%%%%%%%%%%%%%%%%%%%%%%%%%%%%%%%%%%%%%%%%%%%%%%%%%%%%%%%%%%%%%%%%%%%%%
Katedra \makebox[9.5cm][l]{\textbf{}} je saglasna sa predlozhenom temom.
\vspace{-3mm}
\hspace*{13mm} \rule[2.3cm]{9.5cm}{.05mm}\\
\vspace{-1cm}
%%%%%%%%%%%%%%%%%%%%%%%%%%%%%%%%%%%%%%%%%%%%%%%%%%%%%%%%%%%%%%%%%%%%%%%%%%%%%%%
% POPUNJAVA SEF KATEDRE                                                       %
%%%%%%%%%%%%%%%%%%%%%%%%%%%%%%%%%%%%%%%%%%%%%%%%%%%%%%%%%%%%%%%%%%%%%%%%%%%%%%%
\makebox[6.5cm][c]{} \hfill \makebox[6.5cm][c]{}\\
\rule[4mm]{6.5cm}{.05mm} \hfill \rule[4mm]{6.5cm}{.05mm}\\
\vspace{-5mm}
\makebox[6.5cm][c]{\textit{(shef katedre)}} \hfill \makebox[6.5cm][c]{\textit{(datum odobravanja molbe)}}
\end{flushleft}
\end{document} 